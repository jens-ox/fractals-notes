\RequirePackage[utf8]{inputenc}
\RequirePackage[T1]{fontenc}
\RequirePackage{mathtools}

\RequirePackage{hyperref}
\RequirePackage{amsmath}
\RequirePackage{wrapfig}
\RequirePackage{amsfonts}
\RequirePackage{amssymb}
\let\mathbbalt\mathbb
\RequirePackage{fdsymbol}
\RequirePackage[eqno]{tabfigures}
\RequirePackage{natbib}
\RequirePackage{index}
\RequirePackage{tikz}

\RequirePackage{tcolorbox}

\RequirePackage[ngerman]{babel}

\hypersetup{
    colorlinks,
    linkcolor={red!70!black},
    citecolor={blue!50!black},
    urlcolor={blue!80!black}
}

\InputIfFileExists{glyphtounicode}{\pdfgentounicode=1}{}

% Math declarations
\newcommand{\N}{\mathbbalt{N}}
\newcommand{\Z}{\mathbbalt{Z}}
\newcommand{\Q}{\mathbbalt{Q}}
\newcommand{\R}{\mathbbalt{R}}
\renewcommand{\C}{\mathbbalt{C}}

\newcommand{\ot}{\leftarrow}
\newcommand{\To}{\implies}

\newcommand{\e}{\text{\rmfamily e}}
\renewcommand{\i}{\text{\rmfamily i}}
\renewcommand{\O}{\mathcal{O}}

\let\lim\undefined
\DeclareMathOperator{\lim}{lim}
\let\liminf\undefined
\DeclareMathOperator{\liminf}{lim\,inf}
\let\limsup\undefined
\DeclareMathOperator{\limsup}{lim\,sup}

\let\origphi\phi
\let\phi\varphi
\let\origtheta\theta
% \let\theta\vartheta
\let\origepsilon\epsilon
\let\epsilon\varepsilon
\let\origupepsilon\upepsilon
\let\upepsilon\upvarepsilon

\let\smallfrac\@undefined
\newcommand{\smallfrac}[2]{\ensuremath{\frac{#1}{#2}}}

\newcommand{\term}[1]{\textcolor{red!70!black}{\textbf{#1}}}

% Pfad, an dem Grafiken abgelegt werden
\graphicspath{{assets/images/}}

% schönere Malpunkte
\mathcode`\*="8000
{\catcode`\*\active\gdef*{\cdot}}

% Punkte zu Kommas in Mathe-Umgebungen
\DeclareMathSymbol{.}{\mathord}{letters}{"3B}

% Römische Zahlen
\newcommand{\rom}[1]{%
  \textup{\uppercase\expandafter{\romannumeral#1}}%
}

% Pseudocode-Umgebung
\newenvironment{pseudocode}
  {
    \begin{tcolorbox}[colframe=black!3!white,left=0mm]
    \ttfamily
    \footnotesize
  }
  {
    \end{tcolorbox}
  }

% Frage-Umgebung
\newenvironment{theorembox}
  {
    \begin{tcolorbox}[colback=black!3!white,colframe=black!15!white,coltitle=black,fonttitle=\bfseries,arc=2mm,toprule=0.2mm,bottomrule=0.2mm,leftrule=0.2mm,rightrule=0.2mm,left=1mm,right=1mm,titlerule=0mm,middle=0.5mm,enlarge top by=5mm, enlarge bottom by=5mm]
  }
  {
    \end{tcolorbox}
  }

% Inline-Code
\newcommand{\code}[1]{\lstinline[language=c,basicstyle=\ttfamily]|#1|}



\RequirePackage{graphicx} % allow embedded images
  \setkeys{Gin}{width=\linewidth,totalheight=\textheight,keepaspectratio}
  \graphicspath{{graphics/}} % set of paths to search for images
\RequirePackage{booktabs} % book-quality tables
\RequirePackage{units}    % non-stacked fractions and better unit spacing
\RequirePackage{multicol} % multiple column layout facilities
\RequirePackage{lipsum}   % filler text
\RequirePackage{fancyvrb} % extended verbatim environments
  \fvset{fontsize=\normalsize}% default font size for fancy-verbatim environments

% Standardize command font styles and environments
\newcommand{\doccmd}[1]{\texttt{\textbackslash#1}}% command name -- adds backslash automatically
\newcommand{\docopt}[1]{\ensuremath{\langle}\textrm{\textit{#1}}\ensuremath{\rangle}}% optional command argument
\newcommand{\docarg}[1]{\textrm{\textit{#1}}}% (required) command argument
\newcommand{\docenv}[1]{\textsf{#1}}% environment name
\newcommand{\docpkg}[1]{\texttt{#1}}% package name
\newcommand{\doccls}[1]{\texttt{#1}}% document class name
\newcommand{\docclsopt}[1]{\texttt{#1}}% document class option name
\newenvironment{docspec}{\begin{quote}\noindent}{\end{quote}}% command specification environment