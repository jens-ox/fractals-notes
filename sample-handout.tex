
\documentclass[afourpaper]{tufte-handout}

\title{Fraktale Bildcodierung}
\author{Jens Ochsenmeier}
\date{}

\begin{document}

\maketitle

\begin{abstract}
\noindent
  Inhalt des Vortrags ist
  \begin{enumerate}
    \item eine Übersicht über die bisher erlernten Werkzeuge, die für die theoretischen Grundlagen der fraktalen Bildcodierung benötigt werden,
    \item das Collage-Theorem,
    \item die stetige Abhängigkeit zwischen dem Attraktor eines IFS und den Parametern der Kontraktionen und
    \item die Erläuterung von einfachen fraktalen Bildcodierungen anhand der Quadtree-Methode.
  \end{enumerate}
\end{abstract}

\section{Wiederholung --- bisherige Themen}

\subsection{Metrischer Raum}

Ein \emph{metrischer Raum} ist eine Menge \( X \) mit einer reellwertigen Funktion \( d: X \times X \to \R \). Diese misst den Abstand zwischen zwei Punkten \( x,y \in X \) und muss dazu folgende Axiome erfüllen (\( x,y,z \in X \)):
\begin{enumerate}
  \item \textbf{Symmetrie}: \( d(x,y) = d(y,x) \)
  \item \textbf{Positive Definitheit}: \( 0 \leq d(x,y) \) und \( (d(x,y) = 0 \Rightarrow x = y) \)
  \item \textbf{Dreiecksungleichung}: \( d(x,z) \leq d(x,y) + d(y,z) \)
\end{enumerate}

Mithilfe von Cauchy-Folgen lassen sich nun \emph{vollständige metrische Räume} konstruieren. Insbesondere brauchen wir \emph{kompakte metrische Räume}, eine Teilmenge der vollständigen metrischen Räume.

\subsection{``Raum der Fraktale''}

Für eine gegebene Menge \( X \) bezeichnen wir zunächst die Menge aller kompakten Teilmengen von \( X \) mit \( \mathcal{H}(X) \). Die leere Menge schließen wir hierbei aus.

Um mit \( \mathcal{H}(X) \) arbeiten zu können definieren wir passende Abstandsbegriffe:
\begin{itemize}
  \item \( d(x,A) = \min\left \{ d(x,a) : a \in A \right \} \) (\( x \in X \), \( A \in \mathcal{H}(X) \)) sei der Abstand zwischen einem Punkt \( x \in X \) und einem Element \( A \) von \( \mathcal{H}(X) \).
  \item \( d(A,B) = \max\left \{ d(x,B) : x \in A \right \} \) sei der Abstand zwischen zwei Mengen in \( \mathcal{H}(X) \).
\end{itemize}

Mithilfe dieser Abstandsbegriffe konstruieren wir die \emph{Hausdorff-Distanz} zwischen zwei Mengen \( A,B \in \mathcal{H}(X) \) durch
\begin{equation*}
  h(A,B) \coloneqq \max\left \{ d(A,B),\ d(B,A) \right \}\text{.}
\end{equation*}
Die Hausdorff-Distanz bildet eine Metrik auf \( \mathcal{H}(X) \).

Wir erhalten den \emph{Raum der Fraktale}
\begin{equation*}
  (\mathcal{H}(X), h)\text{.}
\end{equation*}
Für uns ist --- zumindest vorerst --- jede Teilmenge von \( (\mathcal{H}(X), h) \) ein Fraktal. Weiter kann gezeigt werden, dass folgender zentraler Satz über den Raum der Fraktale gilt:

\begin{theorembox}
  \textbf{Satz}: \ital{Vollständigkeit des Raums der Fraktale} \\
  \vspace{1mm}
  Es gilt
  \begin{align*}
    (X,d) &\text{ ist ein vollständiger metrischer Raum} \\
    \Rightarrow (\mathcal{H}(X), h) &\text{ ist ein vollständiger metrischer Raum.}
  \end{align*}
\end{theorembox}

\subsection{Kontraktionen}

Eine \emph{Kontraktion} ist eine spezielle Abbildung \( \phi: X \to X \), für die ein \( c \in [0,1) \) existiert, sodass
\begin{equation*}
  \forall x,y \in X : d(\phi(x), \phi(y)) \bm{\leq} c * d(x,y)\text{.}
\end{equation*}
\( c \) wird hier als \emph{Kontraktionsfaktor} bezeichnet.
Eine Kontraktion zieht eine Menge ``in sich selbst zusammen'' --- mindestens so stark wie eine Skalierung mit festem Skalierungsfaktor \( c < 1 \).

Ein wichtiger Sonderfall einer Kontraktion liegt vor, wenn
\begin{equation*}
  \forall x,y \in X : d(\phi(x), \phi(y)) \bm{=} c * d(x,y)\text{.}
\end{equation*}
Eine solche Kontraktion wird \emph{Ähnlichkeitsabbildung} genannt, \( c \) heißt hier \emph{Kontraktionsverhältnis}.

\subsection{Banachscher Fixpunktsatz}

Ein wichtiges Resultat an dieser Stelle ist der \emph{Banachsche Fixpunktsatz}. Er besagt, dass für eine Kontraktion \( \phi \) (wie oben) und eine iterativ definierte Folge \( {(x_n)}_{n \in \N} \) in \( X \) mit
\begin{equation*}
  x_{n+1} = \phi(x_n)
\end{equation*}
folgende Aussagen gelten:
\begin{enumerate}
  \item Für einen \ital{beliebigen} Startwert \( x_0 \) existiert genau ein \( \widetilde{x} \in X \), sodass \( \phi(\widetilde{x}) = \widetilde{x} \).
  \item Es ist \( \underset{n \to \infty}{\lim} x_n = \widetilde{x} \).
\end{enumerate}

Dieses \( \widetilde{x} \) wird \emph{Fixpunkt} der Kontraktion \( \phi \) genannt.

\subsection{(Affine) Transformationen}

Aus den Vorlesungen über Lineare Algebra sind lineare Transformationen bekannt. Sie bilden Geraden auf Geraden ab und fixieren den Ursprung. Dargestellt werden können sie durch die Multiplikation eines Vektors mit einer Matrix:
\begin{equation*}
  t: \R^2 \to \R^2\text{,} \quad t(x,y) \coloneqq \begin{pmatrix}
    a & b \\ c & d
  \end{pmatrix} \begin{pmatrix}
    x \\ y
  \end{pmatrix}\text{.}
\end{equation*}

Eine (2-dimensionale) \emph{affine Transformation} \( w : \R^2 \to \R^2 \) ist eine Abbildung, die durch
\begin{equation*}
  w(x,y) = \begin{pmatrix}
    a & b \\ c & d
  \end{pmatrix} \begin{pmatrix}
    x \\ y
  \end{pmatrix} + \begin{pmatrix}
    e \\ f
  \end{pmatrix}
\end{equation*}
für feste \( a,b,c,d,e,f \in \R \) beschrieben werden kann. Eine affine Transformation setzt sich also aus einer linearen Transformation und einer Translation zusammen.

Eine praktische und übliche Notation ist
\begin{equation*}
  w(x) = Ax + t \quad \text{mit} \quad x = \begin{pmatrix}
    x_1 \\ x_2
  \end{pmatrix},\ A = \begin{pmatrix}
    a & b \\ c & d
  \end{pmatrix},\ t = \begin{pmatrix}
    e \\ f
  \end{pmatrix}\text{.}
\end{equation*}

\subsection{Konstruktion einer affinen Transformation}

Sind eine Menge und eine affin transformierte Kopie dieser gegeben, so kann eine passende affine Transformation, die die Menge auf ihre Kopie abbildet, konstruiert werden, indem drei Punkte der Menge und jeweils die dazugehörige Kopie ermittelt werden.

Seien \( (x_1, x_2) \), \( (y_1, y_2) \), \( (z_1, z_2) \) solche Punkte in der Menge und \( (\widetilde{x}_1, \widetilde{x}_2) \), \( (\widetilde{y}_1, \widetilde{y}_2) \), \( (\widetilde{z}_1, \widetilde{z}_2) \) die zugehörigen Punkte in der Kopie.

Daraufhin lassen sich folgende Gleichungen lösen:
\begin{align*}
  \begin{pmatrix}
    a & b \\ c & d
  \end{pmatrix} \begin{pmatrix}
    x_1 \\ x_2
  \end{pmatrix} + \begin{pmatrix}
    e \\ f
  \end{pmatrix} &= \begin{pmatrix}
    \widetilde{x}_1 \\ \widetilde{x}_2
  \end{pmatrix} \\
  \begin{pmatrix}
    a & b \\ c & d
  \end{pmatrix} \begin{pmatrix}
    y_1 \\ y_2
  \end{pmatrix} + \begin{pmatrix}
    e \\ f
  \end{pmatrix} &= \begin{pmatrix}
    \widetilde{y}_1 \\ \widetilde{y}_2
  \end{pmatrix} \\
  \begin{pmatrix}
    a & b \\ c & d
  \end{pmatrix} \begin{pmatrix}
    z_1 \\ z_2
  \end{pmatrix} + \begin{pmatrix}
    e \\ f
  \end{pmatrix} &= \begin{pmatrix}
    \widetilde{z}_1 \\ \widetilde{z}_2
  \end{pmatrix}
\end{align*}

So lassen sich die 6 Parameter einer zugehörigen affinen Transformation bestimmen.

\subsection{Iterierte Funktionensysteme (IFS)}

Ein \emph{iteriertes Funktionensystem} (IFS) ist \( \left \{ \Phi_1, \dots, \Phi_n \right \} \) --- eine Familie von Kontraktionen auf einem vollständigen metrischen Raum.

Der \emph{Kontraktionskoeffizient} eines solchen IFS ist der größte Kontraktionsfaktor der Kontraktionen \( \Phi_1, \dots, \Phi_n \):
\begin{equation*}
  c = \max\left \{ c_1, \dots,c_n \right \} \quad \text{(\( c_i \coloneqq \) Kontraktionsfaktor von \( \Phi_i \))}
\end{equation*}

Erhält man durch Vereinigung der Bilder der Kontraktionen von einer festen Teilmenge \( C \subset X \) wieder \( C \), so ist C der \emph{Attraktor} des IFS \( \left \{ \Phi_1, \dots, \Phi_n \right \} \).

\section{Modellierung ``echter'' Bilder}

Gegenstand dieses Vortrags ist die Codierung von Bildern mithilfe von Werkzeugen aus der fraktalen Geometrie. Aber was ist ein Bild eigentlich?

Barnsley modelliert solche Bilder entlang von fünf Eigenschaften, die ein Bild haben soll.

\subsection{Die Menge aller echten Bilder}

Grob beschrieben ist ein echtes Bild etwas, dass man irgendwo sehen oder sich zumindest zu sehen vorstellen könnte --- also kein mathematisch greifbares Objekt. Es sei \( \mathcal{R} \) die Menge aller echten Bilder und \( \mathcal{I} \in \mathcal{R} \) ein echtes Bild. Um \( \mathcal{I} \) mathematisch beschreiben und verwenden zu können muss es folgende Eigenschaften erfüllen:

\begin{enumerate}
  \item Es besitzt einen \emph{Träger} \( \largesquare = [a,b] \times [c,d] \subset \R^2 \) und die \emph{Maße} \( (b-a) \) und \( (d-c) \). Der Träger kann beispielsweise eine Tafel oder ein Blatt Papier sein, die Maße sind in einer festen Einheit, beispielsweise Meter.

  Durch den Träger wird ein Bild in den mit der Standardmetrik versehenen euklidischen Raum eingebettet, wodurch die bekannten Werkzeuge aus Geometrie und Topologie zur Verfügung stehen.

  \item Es besitzt gewisse chromatische Werte. Ein chromatischer Wert ist eine Funktion
  \begin{equation*}
    c: \largesquare \to \R\text{,}
  \end{equation*}
  die jedem Punkt des Bildes einen reellwertigen Wert zuordnet. Wir werden in diesem Vortrag nur Bilder in Graustufen betrachten, deswegen reicht für uns ein chromatischer Wert; die Helligkeit an einem bestimmten Punkt.

  \item Einem Bild \( \mathcal{I} \in \mathcal{R} \) ist \ital{keine} Auflösung zugeordnet. Das heißt, dass es für beliebige, endliche Auflösungen beschrieben werden kann, und diese Beschreibungen dasselbe \( \mathcal{I} \in \mathcal{R} \) beschreiben. Die Rasterung findet separat statt.

  \item \( \mathcal{R} \) ist unter Zuschneiden von Bildern abgeschlossen. Das bedeutet, dass ein fester Bereich \( \widetilde{\mathcal{I}} \) eines \( \mathcal{I} \in \mathcal{R} \) ein eigenständiges Bild und somit \( \in \mathcal{R} \) ist.

  \item \( \mathcal{R} \) ist unter invertierbaren affinen Transformationen abgeschlossen. Das bedeutet, dass kein \( \mathcal{I} \in \mathcal{R} \) beliebig vergrößert, verkleinert, gestreckt, gespiegelt, gestaucht und neu positioniert werden kann, sodass die Transformation nicht in \( \mathcal{R} \) liegt.
\end{enumerate}

\subsection{Digitalisierung und Rasterung}

Digitalisierung und Rasterung werden verwendet, um ein \( \mathcal{I} \in \mathcal{R} \) auf Pixel abbilden zu können. Hierzu werden Modelle aus verschiedenen Bereichen der Mathematik, insbesondere aus der Maßtheorie verwendet werden, welche in diesem Vortrag nicht näher besprochen werden sollen.

\section{Collage-Theorem}

Wir haben in anderen Vorträgen gesehen, dass mithilfe eines IFS komplexe Bilder mit nur wenigen Kontraktionen beschrieben werden können.

Idee ist nun, statt ein Bild aus einem vorgegebenen IFS zu generieren, ein möglichst passendes IFS für ein gegebenes Bild zu finden. Offensichtlich wird ein gegebenes Bild in der Regel keine erkennbare selbstähnliche Struktur besitzen. Das \emph{Collage-Theorem} erlaubt die Konstruktion von iterierten Funktionensystemen, deren Attraktoren einem gegebenen Bild möglichst genau entsprechen.

\begin{theorembox}
  \textbf{Satz}: \ital{Collage-Theorem} (Barnsley, 1985) \\
  \vspace{1mm}
  Sei \( (X,d) \) ein vollständiger metrischer Raum, \( L \in \mathcal{H}(X) \), \( \epsilon \geq 0 \).

  Für ein IFS \( \left \{ \Phi_1,\dots,\Phi_n \right \} \) mit Kontraktionskoeffizient \( 0 \leq s < 1 \) und
  \begin{align*}
    h\left( L,\bigcup_{i=0}^n \Phi_i(L) \right) \leq \epsilon \quad &\text{(\( h \): Hausdorff-Metrik)}
    \intertext{gilt:}
    h(L,A) \leq \frac{\epsilon}{1 - s} \quad &\text{(\( A \): Attraktor des IFS).}
  \end{align*}
\end{theorembox}

Um ein gegebenes Bild \( L \in \mathcal{H}(X) \) durch den Attraktor eines IFS anzunähern, muss die Vereinigung (``Collage'') der Kontraktionen des IFS, angewandt auf das gegebene Bild \( L \), dem gegebenen Bild ähneln (im Sinne der Hausdorff-Metrik).

\begin{proof}{}
  Wir zeigen, dass in einem vollständigen metrischen Raum \( (X, d) \) für eine Kontraktion \( f: X \to X \) mit einem Kontraktionsfaktor \( 0 \leq s < 1 \) und einem Fixpunkt \( \widetilde{x} \in X \) gilt:
  \begin{equation*}
    d(x,\widetilde{x}) \leq \frac{d(x,f(x))}{1-s}\text{.}
  \end{equation*} 
  Wir nutzen dazu die Stetigkeit der Abstandsfunktion. Es gilt also
  \begin{align*}
    d(x,\widetilde{x}) &= d\left( x, \underset{n \to \infty}{\lim} f^n(x) \right) = \underset{n \to \infty}{\lim}d(x,f^n(x)) \\
    &\leq \underset{n \to \infty}{\lim}\sum_{m = 1}^n d(f^{m-1}(x),f^m(x)) \\
    &\leq \underset{n \to \infty}{\lim} d(x,f(x))(1+s+\cdots+s^{n-1}) \\
    &\leq \frac{d(x,f(x))}{1-s}\text{.} \qed
  \end{align*}
\end{proof}

\section{Stetige Abhängigkeit von Parametern}

Ein weiterer wichtiger Satz zur Komprimierung von Bildern ist der Satz über die stetige Abhängigkeit des Attraktors eines IFS von den Parametern des IFS.\@ Diese stetige Abhängigkeit wird uns erlauben, durch kleine Änderungen der Parameter kleine Änderungen am Attraktor eines IFS provozieren zu können. Eine praktische Folgerung ist außerdem, dass zwischen verschiedenen Attraktoren ein stetiger Übergang durch Parameteränderungen konstruiert werden kann --- diese Eigenschaft ist besonders praktisch zur Konstruktion und Komprimierung von animierten Bildern und Filmmaterial.

\begin{theorembox}
  \textbf{Stetige Abhängigkeit von Parametern --- Fixpunkt} \\
  \vspace{1mm}
  Es seien \( (P, d_P) \) ein metrischer und \( (X,d) \) ein vollständiger metrischer Raum, außerdem sei \( \Phi : P \times X \to X \) eine Familie von Kontraktionen auf \( X \) mit Kontraktionskoeffizient \( 0 \leq s < 1 \). Für jedes \( p \in P \) ist also \( \Phi(p, \cdot) \) eine Kontraktion auf \( X \).

  \vspace{1em}
  Für jedes \( x \in X \) sei \( \Phi(\cdot, x) \) stetig. Dann ist der Fixpunkt \( \widetilde{x}(p) \) von \( \Phi \) stetig abhängig von \( p \).

  \vspace{1em}
  Das heißt: \( \widetilde{x} : P \to X \) ist stetig.
\end{theorembox}

\begin{proof}{}
  Wir verwenden folgende Notationsvereinfachungen:
  \begin{equation*}
    \widetilde{x}_p \coloneqq \widetilde{x}(p), \quad \Phi_p(x) \coloneqq \Phi(p,x)\text{.}
  \end{equation*}
  Es seien \( p \in P \) und \( \epsilon > 0 \) gegeben. Dann gilt für alle \( q \in P \):
  \begin{align*}
    d(\widetilde{x}_p,\widetilde{x}_q) &= d(\Phi_p(\widetilde{x}_p), \Phi_q(\widetilde{x}_q)) \\
      &\leq d(\Phi_p(\widetilde{x}_p), \Phi_q(\widetilde{x}_p)) + d(\Phi_q(\widetilde{x}_p), \Phi_q(\widetilde{x}_q)) \\
      &\leq d(\Phi_p(\widetilde{x}_p), \Phi_q(\widetilde{x}_p)) + s * d(\widetilde{x}_p, \widetilde{x}_q)\text{.}
  \end{align*}
  Das impliziert, dass
  \begin{equation*}
    d(\widetilde{x}_p, \widetilde{x}_q) \leq \frac{d(\Phi_p(\widetilde{x}_p), \Phi_q(\widetilde{x}_p))}{1-s}\text{.}
  \end{equation*}
  Geht \( p \) gegen \( q \), dann geht die rechte Seite des Terms aufgrund der Folgenstetigkeit gegen \( 0 \).
  \begin{equation*}
    \qed
  \end{equation*}
\end{proof}

Sinnvoll wird diese Erkenntnis allerdings erst, wenn wir die stetige Abhängigkeit zwischen den IFS-Parametern und dem tatsächlichen Attraktor des IFS zeigen können.

\begin{theorembox}
  \textbf{Stetige Abhängigkeit von Parametern --- Attraktor} \\
  \vspace{1mm}
  Es sei \( (X,d) \) ein vollständiger metrischer Raum und \( \left \{ \Phi_1, \dots, \Phi_n \right \} \) ein IFS mit Kontraktionskoeffizient \( s \), wobei \( \Phi_i \) von \( p \in (P, d_P) \) im Sinne der Einschränkung
  \begin{equation*}
    d(\Phi_{i,p}(x), \Phi_{i,q}(x)) \leq k * d_P(p,q)
  \end{equation*}
  für beliebiges \( x \) abhänge.

  \vspace{1em}
  Dann ist der Attraktor \( A_p \in \mathcal{H}(X) \) stetig von \( p \) mittels der Hausdorff-Metrik \( h(d) \) abhängig.
\end{theorembox}

Existiert ein \( C \in \R \) derart, dass
\begin{equation*}
  d(\Phi_p(x), \Phi_q(x)) \leq C * d(p,q)
\end{equation*}
für alle \( p,q \in P \) und \( x \in X \), so können wir im letzten Schritt des vorhergegangenen Beweises folgende Abschätzung verwenden:
\begin{align*}
  d(\widetilde{x}_p, \widetilde{x}_q) \leq \frac{C * d(p,q)}{1-s}\text{.}
\end{align*}

Die Existenz eines solchen \( C \) entspricht der Definition von Lipschitz-Stetigkeit, deswegen schränken wir die Kontraktionen der IFS auf Lipschitz-stetige Kontraktionen ein --- das bedeutet, dass für ein mit \( p \in P \) parametrisiertes IFS
\begin{equation*}
  \left \{ \Phi_{1p},\dots,\Phi_{np} \right \} \quad \text{(\( \Phi_{ip} \) stetig)}
\end{equation*}
ein \( k \in \R_{> 0} \) existiert, sodass gilt:
\begin{equation*}
  d(\Phi_{ip}(x), \Phi_{iq}(x)) \leq k * d_P(p,q)\text{.}
\end{equation*}

Lipschitz-Stetigkeit ist zwar nicht die allgemeinste Forderung, aber sie ist einfach zu handhaben und reicht für unsere Zwecke aus.

Unser Ziel ist nun, für beliebiges \( B \in \mathcal{H}(X) \) nachzuweisen, dass gilt:
\begin{equation*}
  h(\Phi_{i,p}(B), \Phi_{i,q}(B)) \leq k * d_P(p,q)\text{.}
\end{equation*}

Das bedeutet, dass der ``Unterschied'' zwischen den Bildern von \( B \) unter zwei verschieden parametrierten Kontraktionen nach oben durch ein Vielfaches des Abstands der Parameter abgeschätzt werden kann.

\begin{proof}{}
  Diese Aussage kann leicht nachgewiesen werden, indem die Eigenschaften einer Metrik nachgegangen werden:
  \begin{equation*}
    h(\Phi_p(B), \Phi_q(B)) = \max\left \{ d(\Phi_p(B), \Phi_q(B)), d(\Phi_q(B), \Phi_p(B)) \right \}
  \end{equation*}
  mit
  \begin{align*}
    d(\Phi_p(B), \Phi_q(B)) &= \max_{x \in \Phi_p(B)}(d(x, \Phi_q(B)))\text{,} \\
    d(x,\Phi_q(B)) &= \min_{y \in \Phi_q(B)}(d(x,y))\text{.}
  \end{align*}
  Ist \( x \in \Phi_p(B) \), so bedeutet das, dass es ein \( x' \in B \) gibt, sodass \( x = \Phi_p(x') \). Außerdem gilt dann, dass \( \Phi_q(x') \in \Phi_q(B) \). Es folgt
  \begin{equation*}
    d(x, \Phi_q(x')) \leq k * d_P(p,q) \Rightarrow \min_{y \in \Phi_q(B)}(d(x,y)) \leq d(x, \Phi_q(x')) \leq k * d_P(p,q)\text{.}
  \end{equation*}
  Wir erhalten also
  \begin{equation*}
    d(\Phi_p(B), \Phi_q(B)) \leq k * d_P(p,q)\text{.}
  \end{equation*}
  Die Argumentation für \( d(\Phi_q(B), \Phi_p(B)) \) ist analog, wir haben also
  \begin{equation*}
    h(\Phi_p(B), \Phi_q(B)) \leq k * d_P(p,q)\text{.}
  \end{equation*}
  Für endliches \( n \) erhalten wir ein endliches IFS \( \left \{ \Phi_{1_p},\dots,\Phi_{n,p} \right \} \) mit zugehörigen Konstanten \( k_1,\dots,k_n \). Wähle
  \begin{equation*}
    k \coloneqq \max\left \{ k_1,\dots,k_n \right \}\text{.}
  \end{equation*}
  und somit erhalten wir letztendlich
  \begin{equation*}
    h(\Phi_{i,p}(B),\Phi_{i,q}(B)) \leq k * d_P(p,q)\text{.}\qed
  \end{equation*}
\end{proof}
Die Vereinigung solcher Bilder kann sich von Parameter zu Parameter offensichtlich nicht stärker verändern als die maximale Hausdorff-Distanz von oben. Es gilt also
\begin{equation*}
  h\left( W_p(B), W_q(B) \right) \leq k * d_P(p,q)\text{,}
\end{equation*}
wobei
\begin{equation*}
  W_p(B) \coloneqq \bigcup_{i=1}^n \Phi_{i,p}(B)\text{,}
\end{equation*}
\( W_q(B) \) respektive. Wenden wir nun das Resultat der stetigen Abhängigkeit von Fixpunkten an, so erhalten wir
\begin{equation*}
  h(A_p,A_q) \leq \frac{h(A_p, W_q(A_p))}{1-s} \leq k * \frac{d_P(p,q)}{1-s} \quad \text{(Attraktor \( A_p \), abhängig von Parameter \( p \)).}
\end{equation*}

Damit ergibt sich der Satz.

\section{Bildkomprimierung}

\subsection{Warum Bildkomprimierung?}

Die Gründe für Bildkomprimierung sind schnell erklärt: Durch sie wird die Speicherung von Bildern wesentlich kostengünstiger und das Übertragen von Bildern geht schneller. Vorteil hierbei ist, dass das menschliche Auge unempfindlich gegenüber einer Vielzahl von Arten des Informationsverlustes ist.

\subsection{Fraktale Bildkomprimierung}

Grundgedanke der fraktalen Bildkomprimierung ist folgender: Das Bild des Farns von vorhin kann vollständig durch die Angabe der Parameter für die drei affinen Transformationen angegeben und reproduziert werden --- bei beliebiger Auflösung. Das bedeutet, dass nur 18 Zahlen (6 pro Transformation), also ungefähr 18 Byte, gespeichert werden müssen. Um das Bild des Farns auf einer pro-Pixel-Basis zu speichern bräuchte man ein Bit pro Pixel, also 1 Byte für 8 Pixel. Schon für eine geringe Auflösung (200 * 400 Pixel) bräuchte man also 10 Kilobyte --- über 500 mal mehr als für die fraktale Darstellung.

Wir wissen bereits, dass man für gegebene Parameter einfach das zugehörige Fraktal generieren kann. Die Frage ist nun, wie man für ein gegebenes Bild eine Menge affinen Transformationen bestimmen kann, mithilfe welcher sich das Bild hinreichend gut generieren lässt.

Problem echter Bilder ist, dass sie in aller Regel nicht die selbstähnlichen Eigenschaften von Fraktalen haben. Allerdings lassen sich Teile des Bildes finden, welche sich mithilfe von affinen Transformationen ähneln.

\subsection{Partitionierte IFS}

Die Frage ist nun, wie sich diese Eigenschaft mathematisch ausnutzen lässt. Dazu müssen wir die affinen Transformationen der bereits bekannten IFS um zwei Funktionen ergänzen:

\begin{enumerate}
  \item Eine neue Vorschrift adaptiert pro Transformation die Helligkeit.
  \item Eine Maske legt pro Transformation fest, welcher Teil des Urbilds abgebildet wird.
\end{enumerate}

Das heißt, dass nun Teilbereiche des Bilds separat transformiert werden.

Formal ist eine Transformation eines PIFS

\begin{equation*}
  w_i: \square \supset D_i \to w_i(D_i) \eqqcolon R_i \subset \square, \quad w_i(x,y,z) = \begin{pmatrix}
    a_i & b_i & 0 \\
    c_i & d_i & 0 \\
    0 & 0 & g_i
  \end{pmatrix} * \begin{pmatrix}
    x \\ y \\ z
  \end{pmatrix} + \begin{pmatrix}
    e_i \\ f_i \\ h_i
  \end{pmatrix}\text{,}
\end{equation*}

wobei \( g \) und \( h \) die ``neuen'' Parameter sind --- \( g \) ist für den Kontrast und \( h \) für die Helligkeit zuständig.

Damit das gewünschte Bild \( \mathcal{I} \in \mathcal{H}(X) \) Attraktor des PIFS ist fordern wir
\begin{equation*}
  \bigcup_{i=1}^n R_i = \mathcal{I}
\end{equation*}
und
\begin{equation*}
  i \neq j \Rightarrow R_i \cap R_j = \varnothing\text{.}
\end{equation*}

\subsection{PIFS bestimmen}

Man kann die Parameter eines PIFS beispielsweise folgendermaßen bestimmen:

\begin{enumerate}
  \item Unterteile das gegebene Bild in nicht-überlappende Quadrate (8*8 Pixel) \( R \coloneqq \left \{ R_1,\dots, R_n \right \} \).
  \item Unterteile das gegebene Bild in überlappende Quadrate (16*16 Pixel) \( D \coloneqq \left \{ D_1, \dots, D_m \right \} \).
  \item Durchsuche für jede der 8 möglichen Transformationen eines \( R_i \) alle \( D \) nach einem \( D_j \), sodass sich \( R_i \) und \( D_j \) unter der gegebenen Transformation so sehr ähneln wie möglich.
  \item Passe für jede Transformation Belichtung und Kontrast an.
\end{enumerate}

Resultat dieses Algorithmus ist ein PIFS, bestehend aus Transformationen für jedes \( R_i \in R \).

\subsection{Verbesserungen}

Dieser Quadtree-basierte Algorithmus ist zwar relativ naiv, erzielt allerdings erstaunlich gute Ergebnisse. Er lässt sich deutlich verbessern, indem man

\begin{itemize}
  \item mehr verschieden große Quadrate vergleicht und
  \item andere geometrische Formen als Quadrate wählt.
\end{itemize}


\section{Quellen}

\begin{itemize}
  \item Barnsley --- \textbf{Fractals Everywhere} (1993)
  \item Barnsley, Hurd --- \textbf{Fractal Image Compression} (1993)
  \item Fisher --- \textbf{Fractal Image Compression} (1992)
\end{itemize}

\end{document}
